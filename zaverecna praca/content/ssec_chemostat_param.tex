Hlavné premenné, ktoré opisujú mikrobiálne procesy v prírode sú uvedené v Tabuľke \ref{tab: 1}.

\textit{Množstvo mikroorganizmov} môže byť vyjadrené ako biomasa $x$ alebo počet buniek $N$ pri jednobunkových organizmoch (baktérie, kvasinky, spóry) na jednotku pôdy, množstva vody, objemu alebo obsahu plochy. Vláknité organizmy (huby, aktinomycéty) sú charakterizované dĺžkou mycélia $L$ a počtom hýf $n$. Treba zdôrazniť, že $n$ nie je totožné s $N$, pretože vetvenie hýf skôr pripomína delenie buniek pri jednobunkových organizmoch. Vzťah medzi $x$, $N$ a $L$ nie je jednoznačný pretože hmota jednotlivých buniek a šírka hýf sa líši v závislosti od organizmu a podmienok rastu. Všeobecne možno povedať, že pri nadbytku výživných zlúčenín sa formujú veľké bunky resp. široké hýfy, zatiaľ čo pri hladovaní sa tvoria skôr menšie bunky alebo užšie hýfy. Výber biomasy $x$ alebo počet buniek $N$ alebo dĺžku mycélia $L$ závisí na danom prípade. Biomasa $x$ má očividnú výhodu pri skúmaní cyklu uhlíka a živín, zatiaľ čo počet buniek $N$ sa preferuje pri skúmaní populácie napr. výskyt mutácií alebo prenos plazmidov \cite{ref2}.

\begin{table}
	\centering
	\caption{Prehľad hlavných dynamických parametrov opisujúcich biochemický reaktor \cite{ref2}.}
	\label{tab: 1}
	\begin{tabular}{p{5cm} p{1.9cm} p{4cm}}
		\hline
		\textbf{Parameter} & \textbf{Symbol} & \textbf{Rozmer} \\ 
		\hline
		Hustota/koncentrácia biomasy & $x$ & \micg bunkovej hmoty na \gram pôdy; \gram bunkovej hmoty \unitfrac{1}{\squaremeter}; \micg bunkovej hmoty na \ml vody\\
		Počet buniek & $N$ & $10^{6}$ buniek na \gram pôdy; $10^{6}$ buniek na \ml vody\\
		Dĺžka mycélia & \liter & \meter na \gram pôdy; \meter na \ml vody\\
		Počet hýf & $n$ & $10^{6}$ na \gram pôdy; $10^{6}$ na \ml vody\\
		Koncentrácia limitujúceho substrátu & $s$ & \milligram na \gram pôdy; \unitfrac{\gram}{\squaremeter};\unitfrac{\gram}{\liter vody}\\
		Koncentrácia produktu & $p$ & \milligram na \gram pôdy; \unitfrac{\gram}{\squaremeter}; \unitfrac{\gram}{\liter} vody\\
		\hline	
	\end{tabular}
\end{table}

\textit{Koncentrácia limitujúceho substrátu} vo vode alebo v pôde, $(s)$, predstavuje množstvo esenciálnej živiny využívanej mikroorganizmami na rast a rozmnožovanie. Bežne nevieme posúdiť všetky potenciálne dostupné živiny a zamerať sa iba na jednu alebo zopár individuálnych zlúčenín alebo triedu molekúl, ktorá reprezentuje limitujúcu zlúčeninu, pretože chemoorganotrofné mikroorganizmy čerpajú energiu z organických zlúčenín, zatiaľ čo fotosyntetizujúce mikroorganizmy vyžadujú prísun svetla a zdroj fosforu, dusíka a železa \cite{ref2}.

\textit{Množstvo produktov} $(p)$. Sem patria všetky medziprodukty a konečné produkty metabolizmu mikroorganizmov, ktoré vznikajú počas rastu. Typickými medziproduktmi sú organické kyseliny vznikajúce počas glykolýzy. Jediný konečný produkt aeróbnej mikrobiálnej dekompozície je oxid uhličitý, avšak pri anaeróbnych podmienkach vznikajú rôzne organické kyseliny, alkoholy, ketóny atď. \cite{ref2}.
