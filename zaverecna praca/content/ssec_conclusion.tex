Táto práca sa zaoberala problematikou modelov prietokového biochemického reaktora. Na začiatku sme ozrejmili, čo vlastne biochemický reaktor je, aké funkcie plní a vysvetlili sme rozdiely medzi jednotlivými typmi. Odvodili sme najjednoduchší model biochemického reaktora, tzv. Monod model, ktorý sme následne doplnili o časť tvorby produktu. Následne sme sa zaoberali rozdielmi v dynamike a stabilite Monod modelu a modelu s inhibíciou, ktoré boli ilustrované pomocou fázových diagramov. Najväčšiu časť našej pozornosti sme však venovali odhadu parametrov modelu biochemického reaktora. Tu sme porovnávali rôzne prístupy k odhadu parametrov a jednotlivé optimalizačné metódy. Z výsledkov môžeme tvrdiť, že zatiaľ čo odhad parametrov pomocou aproximácie derivácie vykazoval mnohé nedostatky (napr. citlivosť na šum merania, musíme mať k dispozícií časový priebeh koncentrácie substrátu a aj biomasy, s čím súvisia ďalšie problémy atď.), odhad parametrov na základe diferenciálnych rovníc ich odstránil. Avšak náročnosť na výpočet sa tým výrazne zvýšila. Z optimalizačných metód sa ako najvhodnejšia javila simplexová metóda (viď Tabuľku \ref{tab: 4}). Táto dokázala optimalizačný problém vyriešiť efektívnejšie v porovnaní s metódou Luus-Jaakola. Na samotný záver sme uviedli problematiku štruktúry modelu, kde vhodným riešením tohto problému sa javia byť tzv. hybridné modely. 
