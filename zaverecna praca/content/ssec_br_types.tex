Na základe spôsobu prevádzky môže byť biochemický reaktor klasifikovaný ako vsádzkový, kontinuálny a polovsádzkový. 
Pri vsádzkovom spôsobe sa sterilné kultivačné médium naočkuje mikroorganizmami. Počas tohto reakčného obdobia sa s časom menia množstvá buniek, substrátu vrátane výživných solí, vitamínov a produktov. Fermentácia prebieha vopred stanovenú dobu a produkt sa zozbiera na konci.

V polovsádzkovom režime sa do reaktora postupne pridávajú živiny, ako prebiehajú biochemické reakcie, aby sa získali lepšie výťažky a vyššia selektivita spolu s reguláciou reakčnej teploty. Produkty sa zbierajú na konci výrobného cyklu ako pri vsádzkovom biochemickom reaktore.Charakteristickou črtou kontinuálneho biochemického reaktora je proces neustáleho dodávania substrátu. Prúd kvapaliny alebo suspenzie sa kontinuálne privádza a odstraňuje z reaktora. Na dosiahnutie rovnomerného zloženia a teploty je potrebné mechanické alebo hydraulické miešanie. Kultivačné médium, ktoré je buď sterilné alebo obsahuje mikroorganizmy, sa nepretržite dodáva do biochemického reaktora, aby sa udržal stabilný stav. Reakčné premenné a kontrolné parametre zostávajú konzistentné a vytvárajú v reaktore časovo konštantný stav. Výsledkom je nepretržitá produktivita.

Tradičné vsádzkové miešacie tankové reaktory (STR) a kontinuálne miešané tankové reaktory (CSTR) existujú už po stáročia a sú stále široko prijímané v chemickom a biologickom priemysle kvôli ich jednoduchosti. Ostatné biochemické reaktory, ktoré majú špeciálne konštrukčné a prevádzkové vlastnosti sú foto-bioreaktory, rotačné bubnové reaktory, hmlový, membránový biochemický reaktor, reaktory s náplňou a fluidnou vrstvou atď. Tieto boli navrhnuté tak, aby vyhovovali špecifickým procesom \cite{ref1}.
