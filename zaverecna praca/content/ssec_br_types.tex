Na základe spôsobu prevádzky môže byť bioreaktor klasifikovaný ako vsádzkový, kontinuálny a polovsádzkový. 
Pri vsádzkovom spôsobe sa sterilné kultivačné médium naočkuje mikroorganizmami. Počas tohto reakčného obdobia sa s časom menia množstvá buniek, substrátu vrátane výživných solí, vitamínov a produktov. Fermentácia prebieha vopred stanovenú dobu a produkt sa zozbiera na konci.
% RP: Tu mi pride, ze by mal zacat novy odsek
V polovsádzkovom réžime sa do reaktora postupne pridávajú živiny, ako
% RP: neopravoval/nekomentoval som uz ``réžime'' v predchadzajucej verzii? 
prebiehajú bioreakcie, aby sa získali lepšie výťažky a vyššia selektivita spolu s reguláciou reakčnej teploty. Produkty sa zbierajú na konci výrobného cyklu ako pri vsádzkovom bioreaktore.
Charakteristickou črtou kontinuálneho bioreaktora je proces neustáleho dodávania substrátu. Prúd kvapaliny alebo suspenzie sa kontinuálne privádza a odstraňuje z reaktora. Na dosiahnutie rovnomerného zloženia a teploty je potrebné mechanické alebo hydraulické miešanie. Kultivačné médium, ktoré je buď sterilné alebo obsahuje mikroorganizmy, sa nepretržite dodáva do bioreaktora, aby sa udržal stabilný stav. Reakčné premenné a kontrolné parametre zostávajú konzistentné a vytvárajú v reaktore časovo konštantný stav. Výsledkom je nepretržitá produktivita.
% RP: Tu mi pride, ze by mal zacat novy odsek
Tradičné vsádzkové miešacie tankové reaktory (STR) a kontinuálne miešacie tankové
reaktory (CSTR)
% RP: Moja slovencina v tomto nie je top, ale namiesto ``miešacie'' by sa mi viac hodilo ``miešané''
existujú už po stáročia a sú stále široko prijímané v chemickom a biologickom priemysle kvôli ich jednoduchosti. Ostatné bioreaktory, ktoré majú špeciálne konštrukčné a prevádzkové vlastnosti sú foto-bioreaktory, rotačné bubnové reaktory, hmlový bioreaktor, membránový bioreaktor, bioreaktory s náplňou a fluidnou vrstvou atď. Tieto boli navrhnuté tak, aby vyhovovali špecifickým procesom \cite{ref1}.
