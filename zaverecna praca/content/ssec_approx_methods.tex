Na začiatku by bolo vhodné spomenúť, že nie vždy dokážeme získať analytické riešenie modelu, ktoré by nám veľmi uľahčilo prácu, pretože by šlo o výrazne jednoduchšiu optimalizačnú úlohu regresie dát. V prípade, že nemáme k dispozícii analytické riešenie, je nutné priebehy modelu počítať numerickými metódami, čo nás privádza k rôznym návrhom optimalizačných problémov. 

Spôsob, akým získať neznáme parametre modelu, môže byť rôzny a zavisí najmä od množstva údajov, ktorými disponujeme. V prípade, že z reálneho zariadenia dokážeme získať viaceré časové priebehy, teda merať, môžeme parametre odhadnúť na základe aproximácie derivácie. Vysvetlíme si to na príklade biochemického reaktora. V prípade, že dokážeme merať časový priebeh koncentrácie biomasy a substrátu, môžeme na základe rovnice \eqref{eq:5} odhadnúť parametre $K_{M}$ a $\mu_{m}$. Optimalizačný problém by mohol byť sformulovaný následovne:

\begin{equation}
	\min_{\mu_{m},K_{M}} \quad \sum_{n=1}^{N} \left(\Delta_{i}^m-\mu(s_i^m,\mu_{m},K_M)\right)^2,  
\end{equation}

\noindent kde $ N $ je celkový počet dát, $ \Delta_{i} = x_{i}^m-x_{i-1}^m $ je diferencia dvoch susedných nameraných hodnôt koncentrácie biomasy, $ s_i^m $ je hodnota nameranej koncentrácie substrátu v \textit{i}--tom bode a tvar špecifickej rýchlosti rastu závisí od daného modelu a môže mať tvar daný buď \eqref{eq:3}, alebo \eqref{eq:8}.

Prvý prístup je síce veľmi jednoduchý, ale má svoje nevýhody. Najväčším problémom je však skutočnosť, že bežne meriame iba koncentráciu substrátu a tým pádom prichádzame o značnú časť informácií. Druhý prístup obchádza tento problém a zameriava sa na hľadanie neznámych parametrov samotných diferenciálnych rovníc. Zatiaľ čo návrh optimalizačného problému sa príliš nezmení, zložitosť výpočtu sa výrazne zvýši. V tomto prípade je nutné, v každom nameranom bode, niekoľkokrát numericky vyhodnotiť priebehy koncentrácie biomasy, produktu a aj substrátu a porovnať ich s nameraným signálom  tak, aby sme získali optimálne riešenie. Túto problematiku by sme mohli sformulovať následovne: 

\begin{equation}
	\begin{aligned}
		\min_{\mu_{m},K_{M}} \quad & \sum_{n=1}^{N} \left(s_{i}^m-s_{i}\right)^2, \\
		\textrm{s.t.} \quad & \dot{s} = D(s_{in}-s)-\frac{1}{Y_x}\mu(s,\mu_{m},K_M)x-\frac{1}{Y_p}\nu x \\
							& \dot{x} = (\mu(s,\mu_{m},K_M)-D)x
	\end{aligned}
\end{equation}

kde $ s_{i} $ je koncentrácia substrátu v \textit{i}--tom bode vypočítaná na základe modelu biochemického reaktora.

Ostáva tu však otázka, ako správne voliť štruktúru špecifickej rýchlosti rastu, aby sme dokázali správne určiť odhadované parametre. Túto informáciu nevieme jednoducho získať z nameraných údajov a môžeme dospieť k nesprávnemu modelu.
