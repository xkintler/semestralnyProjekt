Na začiatku by bolo vhodné spomenúť, že nie vždy dokážeme získať analytické riešenie modelu, ktoré by nám veľmi uľahčilo prácu, a preto je nutné priebehy modelu počítať numerickými metódami. 
% RP: Prilis otvorene zakoncenie tejto myslienky/odseku

Spôsob, akým získať neznáme parametre modelu, môže byť rôzny a zavisí najmä od množstva údajov, ktorými disponujeme. 
%
V prípade, že z reálneho zariadenia dokážeme získať viacere časové priebehy, teda merať, môžeme parametre odhadnúť na základe aproximácie derivácie. Vysvetlíme si to na príklade biochemického reaktora. V prípade, že dokážeme merať časový priebeh koncentrácie biomasy a substrátu, môžeme na základe rovnice \ref{eq:5} odhadnúť parametre $K_{M}$ a $\mu_{m}$.
% RP: Nie je to velmi dobre vysvetlenie. Chceli by to bud definiciu optimalizacneho problemu alebo obsirnejsi opis. Ta prva by bola vhodnejsia...nieco podobne aj tak budete potrebovat v prezentacii.

Prvý prístup je síce veľmi jednoduchý, ale má svoje nevýhody. Najväčším problémom je však skutočnosť, že bežne meriame iba koncentráciu substrátu a tým pádom prichádzame o značnú časť informácií. Druhý prístup obchádza tento problém a zameriava sa na hľadanie neznámych parametrov samotných diferenciálnych rovníc. Zatiaľ čo návrh optimalizačného problému sa príliš nezmení, zložitosť výpočtu sa výrazne zvýši. V tomto prípade je nutné, v každom nameranom bode, niekoľkokrát numericky vyhodnotiť priebehy koncentrácie biomasy, produktu a aj substrátu a porovnať ich s nameraným signálom  tak, aby sme získali optimálne riešenie.
% RP: Opat by tu mala byt matematicka formulacia optimalizacneho problemu.

Ostáva tu však otázka, ako správne voliť štruktúru špecifickej rýchlosti rastu, aby sme dokázali správne určiť odhadované parametre. Túto informáciu nevieme jednoducho získať z nameraných údajov a môžeme dospieť k nesprávnemu modelu.
