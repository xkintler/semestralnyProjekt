Biochemický reaktor sa dá vo všeobecnosti definovať ako nádoba, ktorá využíva aktivitu biologického katalyzátora na dosiahnutie požadovanej chemickej premeny \cite{ref1}.
Biochemický reaktor všeobecne poskytuje biomechanické a biochemické prostredie, ktoré riadi prenos živín a kyslíka do buniek a produkty metabolizmu z buniek. Dal by sa tiež označiť ako zariadenie, určené na optimálny rast a metabolickú aktivitu organizmu, pôsobením biokatalyzátora, enzýmu alebo mikroorganizmov a buniek zvierat alebo rastlín. Surovinou môže byť organická alebo anorganická chemická zlúčenina alebo dokonca komplexný materiál. Produkt konverzie môže zahŕňať pekárske kvasinky, proteín, štartovacie kultúry alebo primárne metabolity (napr. aminokyseliny, organické kyseliny, vitamíny, polysacharidy, etanol atď.) a sekundárne metabolity (napr. antibiotiká). Biochemické reaktory sa môžu použiť na biokonverziu alebo biotransformáciu produktov (steroidná biotransformácia, L-sorbitol), enzýmov (amyláza, lipáza, celuláza), rekombinantných produktov (niektoré vakcíny, hormóny, ako je inzulín a rastové hormóny). Biochemický reaktor musí byť navrhnutý tak, aby vyhovoval konkrétnemu procesu \cite{ref1}.

