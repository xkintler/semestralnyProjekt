Pri návrhu optimalizácie, by našim cieľom mohlo byť minimalizovať výrobné náklady alebo maximalizovať efektívnosť výroby. Optimalizačný algoritmus je postup, ktorý sa vykonáva iteratívne porovnávaním rôznych riešení, až kým sa nenájde optimálne alebo uspokojivé riešenie. Dnes existujú dva odlišné typy optimalizačných algoritmov.

\begin{itemize}
	\item[\textbf{(a)}] \textbf{Deterministické algoritmy.} 
	Tieto využívajú špecifické pravidlá na posúvanie riešenia od jedného k druhému. Ide o metódy, ktoré sú založené na poznaní gradientu účelovej funkcie alebo na odhade gradientu. Môže sem patriť napr. metóda poklesu gradientu, Newtonova metóda, simplexova (Nelder--Mead) metóda alebo metóda hľadania extrému.
	\item[\textbf{(a)}] \textbf{Stochastické algoritmy.} 
	Tieto majú povahu pravdepodobnostných pravidiel. Stávajú sa veľmi obľúbenými kvôli určitým vlastnostiam, ktoré deterministické algoritmy nemajú. Sem môžu patriť metódy ako simulované žíhanie alebo metóda Luus--Jaakola.
\end{itemize}

Metódy založené na poznaní gradientu účelovej funkcie vedú k lokálnemu optimu, zatiaľ čo stochastické metódy často vedú ku globálnemu optimu, ale vyžadujú značné množstvo vyhodnotení účelovej funkcie. Z tohto dôvodu, pre rozmerovo väčšie problematiky, sú negradientové metódy menej žiadúce.
