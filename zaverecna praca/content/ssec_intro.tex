Biochemické reaktory sa považujú za dôležitú súčasť chemického priemyslu. Široká škála dôležitých zlúčenín ako farmaceutické produkty, rôzne polyméry alebo produkty potravinárskeho priemyslu sa vyrábajú pomocou určitého fermentačného média (rôzne baktérie, kvasinky, vláknité huby alebo enzýmy) za prísne stanovených podmienok v biochemickom reaktore \cite{SRINIVASAN20031}. Aby sme dokázali zabezpečiť optimálny chod biochemického reaktora z hľadiska ekonomiky, bezpečnosti alebo environmentálneho zaťaženia, potrebujeme disponovať správnym matematickým modelom, ktorý by adekvátne opisoval správanie daného procesu. Aj keď existuje množstvo matematických modelov, ktoré môžu správne popisovať fungovanie biochemického reaktora, správne identifikovať ich parametre, je často omnoho zložitejšia problematika a je esenciálna pre zabezpečenie optimálnych podmienok.

V tejto práci sme sa venovali najmä problematike odhadu parametrov modelov biochemického reaktora a problematike s tým súvisiacou ako napríklad prístupy k odhadu parametrov alebo optimalizačné metódy. Vysvetlili sme si základne vlastnosti biochemického reaktora, na základe ktorých sme odvodili matematický model, ktorý sme podrobili analýze stability a dynamiky.

