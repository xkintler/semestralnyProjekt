Biochemické reaktory sa považujú za dôležitú súčasť chemického priemyslu. Široká škála dôležitých zlúčenín ako farmaceutické produkty, rôzne polyméry alebo produkty potravinárského priemyslu sa vyrábajú pomocou určitého fermentačného média (rôzne baktérie, kvasiky, vláknité huby alebo enzýmy) za prísne stanovených podmienok v biochemickom reaktore. Aby sme dokázali zabezpečiť optimálny chod biochemického reaktora z hľadiska ekonomiky, bezpečnosti alebo enviromentálneho zaťaženia, potrebujeme disponovať správnym matematickým modelom, ktorý by adekvátne opisoval správanie daného procesu. Aj keď existuje množstvo matematických modelov, ktoré môžu správne popisovať fungovanie biochemického reaktora, správne identifikovať ich parametre, je často omnoho zložitejšia problematika a je esenciálna pre zabezpečenie optimálnych podmienok.