Najjednoduchším matematickým modelom, ktorý opisuje prietokový biochemický reaktor je tzv. Monod model. Tento model je veľmi obľúbený kvôli svojej jednoduchosti. Zakladá sa na dvoch predpokladoch: \text{1)} špecifická rýchlosť rastu buniek závisí od koncentrácie substrátu a  \text{2)} tvorba biomasy je spojená so spotrebou substrátu. Formulácia rovníc, ktoré popisujú materialovú bilanciu biomasy je nasledovná:
% RP: spell check ``materialovú''
\begin{table}[H]
% RP: Nie je to chyba alebo problem, ale tu by prostredie table vobec nemuselo byt a stacil by samotny tabular
	\centering
	\begin{tabular}{ccccc}
		akumulácia & & množstvo & & množstvo \\
		bunkovej & = & vzniknutých & -- & odobraných \\
		hmoty & & buniek & & buniek \\
	\end{tabular}
\end{table}
\noindent a pre materialovú bilanciu substrátu platí:
\begin{table}[H]
	\centering
	\begin{tabular}{ccccccc}
		akumulácia & & množstvo & & množstvo & & množstvo\\
		substrátu & = & dodaného & -- & odobraného & -- & spotrebovaného .\\
		v systéme & & substrátu & & substrátu & & substrátu MO\\
	\end{tabular}
\end{table}
\noindent Ak uvažujeme, že objem reaktora sa nemení a prítok substrátu sa rovná odtoku suspenzie, potom môžme písať: 
\begin{align}
	&V\left(\frac{\dd x}{\dd t}\right) = V\mu(s)x - Fx, \label{eq:1} \\
	&V\left(\frac{\dd s}{\dd t}\right) = Fs_{in} - Fs - V\frac{1}{Y_{x}}\mu(s)x. \label{eq:2}
\end{align}

\noindent Ak obe strany rovníc vydelíme objemom reaktora a označíme si pomer $\frac{F}{V} = D$, rýchlosť riedenia, môžme rovnice \ref{eq:1} a \ref{eq:2} upraviť do nasledovného tvaru:
% RP: Rovnice je lepsie referencovat pomocou \eqref
\begin{align} 
	&\frac{\dd x}{\dd t} = \left(\mu(s) - D\right)x, \text{kde}  \quad \mu(s) = \mu_{m}\frac{s}{K_{M} + s}, \label{eq:3} \\
	&\frac{\dd s}{\dd t} = D\left(s_{in} - s\right) - \frac{1}{Y_{x}}\mu(s)x. \label{eq:4}
\end{align}

\begin{table}
	\centering
	\caption{Parametre Monod modelu, ich symbol a rozmer.}
	\label{tab: 2}
	\begin{tabular}{lll}
		\hline
		\textbf{Parameter} & \textbf{Symbol} & \textbf{Rozmer} \\
		\hline
		Špecifická rýchlosť rastu & $\mu(s)$ & \unitfrac{1}{h} \\
		Maximálna špecifická rýchlosť rastu & $\mu_{m}$ & \unitfrac{1}{h} \\
		Michaelisova konštanta & $K_{M}$ & \unitfrac{g}{L} \\
		Výťažok (biomasa) & $Y_{x}$ & \\
		Objem reaktora & $V$ & \unit{L} \\
		Prietok substrátu/suspenzie & $F$ & \unitfrac{L}{h} \\
		Koncentrácia biomasy & $x$ & \unitfrac{g}{L} \\
		Koncentrácia substrátu & $s$ & \unitfrac{g}{L} \\
		Koncentrácia čerstvého substrátu & $s_{in}$ & \unitfrac{g}{L} \\
		\hline
	\end{tabular}
\end{table}

Rovnice \ref{eq:3} a \ref{eq:4} tvoria najjednoduchší opis biochemického reaktora -- Monod model, a význam jednotlivých parametrov je uvedený v Tabuľke \ref{tab: 2}. Avšak, tento model má množstvo nedostatkov. Nedokáže vysvetliť jednotlivé fázy rastu, ktoré sú pozorované experimentálne a to: lag-fázu, smrť buniek na základe hladovania, tvorbu produktu atď. Tieto nedostatky boli doplnené u tzv. štruktorovaných modelov.
% RP: spell check ``štruktorovaných''

Model, ktorý berie do úvahy aj tvorbu produktu, získame doplnením Monod modelu do nasledovného tvaru:
\begin{align} 
&\frac{\dd x}{\dd t} = \left(\mu(s) - D\right)x, \label{eq:5} \\
&\frac{\dd s}{\dd t} = D\left(s_{in} - s\right) - \frac{1}{Y_{x}}\mu(s)x - \frac{1}{Y_{p}}\nu x, \label{eq:6} \\
&\frac{\dd p}{\dd t} = \nu x - Dp, \label{eq:7}
\end{align}
kde $p$ predstavuje koncentráciu produktu v $gL^{-1}$,
% RP: Tu uz zrazu nie su pouzite units?
$Y_{p}$ je
bezrozmerný
% RP: Po slovensky je ``bezrozmerový''. ``Rozmerný'' znamena majuci velky rozmer.
koefcient
% RP: spell check ``koefcient''
výťažnosti produktu a $\nu$ predstavuje kinetický člen rýchlosti tvorby produktu v jednotkách času napr. $ h^{-1} $. Do rovnice \ref{eq:4} sme doplnili časť, ktorá vraví, že časť substrátu sa spotrebuje na tvorbu produktu a rovnica \ref{eq:7} predstavuje obyčajnú materiálovú bilanciu produktu.

Ak by sme chceli do modelu zakomponovať tendenciu úmrtia mikroorganizmov počas hladovania, treba upraviť špecifickú rýchlosť rastu $\mu(s)$ tak, že bude obsahovať inhibičný člen $ K_i $, ktorého rozmer je $gL^{-1}$.
% RP: S tymto nesuhlasim. Inhibicia modeluje situaciu ked sa prilis vysoka koncentracia substratu stane pre mikroorganizmy toxickou. Dzem musi obsahovat 50% cukru prave preto aby sa nepokazil.
Špecifická rýchlosť rastu potom nadobudne tvar:
\begin{equation} \label{eq:8}
	\mu(s) = \mu_{m}\frac{s}{K_{M} + s + \frac{s^2}{K_i}}.
\end{equation}
